\title{Securing Kademlia with a low overhead PKI}
\author{Orson Peters}
\date{\today}

\documentclass[12pt]{article}

\usepackage{enumitem}
\setdescription{leftmargin=\parindent,labelindent=1cm}
\usepackage[parfill]{parskip}

\begin{document}
\maketitle

\begin{abstract}
\noindent This paper outlines a low overhead protocol that creates a
robust public-key infrastructure (PKI) to secure Kademlia.  Presented here is a
centralized solution that uses a certification authority that provides identity
to nodes. If correctly implemented this solution implicitly resists common
attacks aimed at DHTs and can be extended in various ways to further limit
malicious activity to a minimum, depending on the use case.  This solution is
unprecedented in low overhead (both latency and bandwidth) and supporting a
recursive/iterative routing hybrid.
\end{abstract}

\section{Introduction}\label{introduction} Kademlia \cite{kademlia} is a
high-profile Distributed Hash Table (DHT). Used in many P2P applications
(LimeWire, Gnutella, Overnet, EDonkey2000, eMule, BitTorrent, etc) it is the
most used DHT around. For good reason, its nifty topology allows for fast,
flexible and low-overhead routing. However, Kademlia does not concern itself
with security, and is vulnerable to many attacks.

In this paper we expand on the work in \cite{aiello2008tempering}. Similarly,
we propose a centralized certificate authority (CA) trusted by all
participating nodes. However, we reduce overhead by introducing optimizations
such as smaller signatures, timestamps instead of nonces and a novel way to
pick randomized node ids. Through the use of secure request tokens and node ids
we also enable a hybrid between recursive and iterative routing for lower
latency while maintaining security.

First we discuss what attacks Kademlia is vulnerable to. Then we propose the
protocol, and how this differs from previous solutions. Afterwards we discuss
how well the attacks on Kademlia are mitigated and what holes are potentially
still left open. Last, we evaluate the overhead of the system and further
improvements.

\section{Terminology}\label{terminology}

Throughout this paper we use the following terminology:

\begin{description}[noitemsep]
    \item[$a || b$] - the concatenation of $a$ and $b$
    \item[$H(x)$] - the SHA-1 hash of $x$
    \item[$K^+$] - the public key of the Ed25519 key pair $K$
    \item[$K^-$] - the private key of the Ed25519 key pair $K$
    \item[$Sign(x, K)$] - the Ed25519 signature of x signed with the key pair $K$
    \item[$Verify(s, x, K^+)$] - verification of the Ed25519 signature $s$ of message $x$ with $K^+$
\end{description}

Ed25519 \cite{bernstein2011high} is a novel public key signing algorithm, with
the desirable property of having small signatures and public keys (respectively
64 and 32 bytes). Furthermore, it is fast, highly secure and allows
randomization of a public key without knowing the private key (more on this
later). These properties make it ideal for our protocol. 



\section{Attacks}

Kademlia suffers from a wide range of attacks. In this paper
we only discuss and prevent attacks against the network infrastructure - not
network content. We do however provide a provable identity for every node,
allowing various solutions like blacklists and reputation systems to secure
content.

\paragraph{Man In The Middle.} All unsecured internet protocols suffer from
this attack. Because the transport layer is not secure any intermediate hop can
change or drop any packet, and any party might be able to forge packets.

\paragraph{Replay attack.} This is a more subtle version of the Man In The
Middle (MITM) attack. An adversary eavesdrops some secured communication and
stores it. Now if the communication does not contain any temporary resource
such as a timestamp or nonce (number used once), the communication gets
accepted when sent by the adversary - breaking the security. Worse, if the
communication does not address a receiver for the content the adversary may
\emph{replay} the communication to any party. Replay attacks are attacks in
its own right, but may be used to set up other attacks. For example, if a
secured request for content is sent out, but does not contain the intended
receiver for the request an adversary might send this request to many content
providers, effectively creating a cheap and anonymous DDOS attack on the
requester.

\paragraph{Routing attack.} Each Kademlia node keeps track of a number of nodes
to communicate with - the routing table. A routing attack is aimed at this
table, filling it with either invalid, offline or even malicious nodes,
hampering or even stopping the connectivity of the attacked node.

\paragraph{Eclipse attack.} The Eclipse attack \cite{singh2006eclipse} is a
routing attack where malicious nodes exploit the topology of the network to
"surround" a victim node. If successful all or a majority of the routing of the
victim node goes through the malicious nodes, allowing them to manipulate or
block the communication.

\paragraph{Distributed Denial of Service attack.} The Distributed Denial of
Service attack (DDoS) is an attack where many computers flood one computer with
requests, requiring so much resources the entire service is stopped or slowed
down to a crawl.

\paragraph{Sybil attack.} The Sybil attack \cite{douceur2002sybil} works by
having one adversary provide many identities to the network, effectively
gaining power. While it is not an attack on its own it can be used to greatly
increase other attacks in effectiveness.

Kademlia suffers from all attacks described above. Our protocol is designed
to explicitly or implicitly resist all of them.


 
\section{The protocol}

There exists a centralized certificate authority, the
CA, that is known to and trusted by all nodes participating in the network.
Furthermore, everyone knows the CAs public key, $CAK^+$.

Because we use timestamps instead of nonces is most locations in this protocol
the network has a synchronized time, called the network time. While its not
necessary to synchronize the time through the CA, it is the most practical and
easily secured way of doing so - therefore we specify it in the protocol.


\subsection{Synchronizing time}

Before anything, a node should synchronize it's time to the network time. A
node wanting to get the network time must generate a nonce, and send it to the
CA along with some synchronize time opcode. The CA then replies with $nonce,
network\_time, sign(nonce || network\_time, CAK^+)$. The node then verifies the
nonce and the signature. Furthermore, the node must keep track of the time the
request took (the round-trip-time). If this is more than a certain timeout, say
5 seconds, the node must reject the reply. Last, if everything was valid a node
should substract half the round-trip-time from the time as a mean of crude
accuracy improvement.

Because the reply is signed and contains a nonce all active MITM attacks and
replay attacks are futile. Because the node rejects old replies a MITM can not
delay a response in an attempt to desynchronize the network time. Assuming the
worst case where a MITM maximized the delay the network time is desynchronized
by at most $timeout/2$ seconds, which is very acceptable.


\subsection{Getting a certificate}

After synchronizing to the network time a node needs a certificate from the CA
in order to be able to participate in the network. The CA signs the nodes
identity, as well as a public key of the node, allowing the node to
subsequently prove ownership of the certificate to peers.

First the node must generate a new public/private keypair.


\bibliographystyle{ieeetr}
\bibliography{paper}

\end{document}
